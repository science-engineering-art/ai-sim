%%%%%%%%%%%%%%%%%%%%%%%%%%%%%%%%%%%%%%%%%
% Journal Article
% LaTeX Template
% Version 1.4 (15/5/16)
%
% This template has been downloaded from:
% http://www.LaTeXTemplates.com
%
% Original author:
% Frits Wenneker (http://www.howtotex.com) with extensive modifications by
% Vel (vel@LaTeXTemplates.com)
%
% License:
% CC BY-NC-SA 3.0 (http://creativecommons.org/licenses/by-nc-sa/3.0/)
%
%%%%%%%%%%%%%%%%%%%%%%%%%%%%%%%%%%%%%%%%%

%----------------------------------------------------------------------------------------
%	PACKAGES AND OTHER DOCUMENT CONFIGURATIONS
%----------------------------------------------------------------------------------------

%\documentclass[10pt]{article} % Single column

\documentclass[colorinlistoftodos,twoside,twocolumn]{article} % Two column

\usepackage{blindtext} % Package to generate dummy text throughout this template 

\usepackage[sc]{mathpazo} % Use the Palatino font
\usepackage[T1]{fontenc} % Use 8-bit encoding that has 256 glyphs
\linespread{1.05} % Line spacing - Palatino needs more space between lines
\usepackage{microtype} % Slightly tweak font spacing for aesthetics

\usepackage[spanish]{babel} % Language hyphenation and typographical rules

\usepackage[hmarginratio=1:1,top=32mm,columnsep=20pt]{geometry} % Document margins
\usepackage[hang, small,labelfont=bf,up,textfont=it,up]{caption} % Custom captions under/above floats in tables or figures
\usepackage{booktabs} % Horizontal rules in tables

\usepackage{lettrine} % The lettrine is the first enlarged letter at the beginning of the text

\usepackage{enumitem} % Customized lists
\setlist[itemize]{noitemsep} % Make itemize lists more compact

\usepackage{abstract} % Allows abstract customization
\renewcommand{\abstractnamefont}{\normalfont\bfseries} % Set the "Abstract" text to bold
\renewcommand{\abstracttextfont}{\normalfont\small\itshape} % Set the abstract itself to small italic text

\usepackage{titlesec} % Allows customization of titles
\renewcommand\thesection{\Roman{section}} % Roman numerals for the sections
\renewcommand\thesubsection{\roman{subsection}} % roman numerals for subsections
\titleformat{\section}[block]{\large\scshape\centering}{\thesection.}{1em}{} % Change the look of the section titles
\titleformat{\subsection}[block]{\large}{\thesubsection.}{1em}{} % Change the look of the section titles

\usepackage{fancyhdr} % Headers and footers
\pagestyle{fancy} % All pages have headers and footers
\fancyhead{} % Blank out the default header
\fancyfoot{} % Blank out the default footer
\fancyhead[C]{Traffic Lights Optimization} % Custom header text
\fancyfoot[RO,LE]{\thepage} % Custom footer text

\usepackage{titling} % Customizing the title section

\usepackage[colorlinks]{hyperref} % For hyperlinks in the PDF

\usepackage{graphicx} % For images

\usepackage{pifont} % bullets

\usepackage{amsmath}


% Keywords command
\providecommand{\keywords}[1]
{
	\small	
	\vspace{0.5em}
	\noindent \textbf{\textit{Palabras clave --- }} #1
}

%----------------------------------------------------------------------------------------
%	TITLE SECTION
%----------------------------------------------------------------------------------------

\setlength{\droptitle}{-4\baselineskip} % Move the title up

\pretitle{\begin{center}\Huge\bfseries} % Article title formatting
	\posttitle{\end{center}} % Article title closing formatting
\title{\normalsize{Proyecto Final Inteligencia Artificial - Simulaci\'on}\\
	\Huge\bfseries Traffic Lights Optimization\\
} % Article title
\author{% 
	\normalsize\textsc{Integrantes:}\\
	\normalsize\textsc{Leandro Rodr\'iquez Llosa}\\
	\normalsize\textsc{Laura V. Riera P\'erez}\\ 
	\normalsize\textsc{Marcos M. Tirador del Riego} \\[2ex]
	\normalsize\textsc{Grupo: C-311} \\[2ex]
	\small Tercer a\~no. Ciencias de la Computaci\'on. \\ % institution
	\small Facultad de Matem\'atica y Computaci\'on, Universidad de La Habana, Cuba \\ % institution
}
\date{\footnotesize Enero 2023 } % Leave empty to omit a date


% Abstract configurations
\renewenvironment{abstract}
{\small
	\begin{center}
		\bfseries \abstractname\vspace{-.5em}\vspace{0pt}
	\end{center}
	\list{}{
		\setlength{\leftmargin}{1.5cm}%
		\setlength{\rightmargin}{\leftmargin}%
	}%
	\item\relax}
{\endlist}


\usepackage{todonotes} % \TODO
\usepackage{listings} % Code listings
\usepackage{xcolor}
\definecolor{backcolour}{rgb}{0.95,0.95,0.92}

\newcommand{\csl}[1]{\colorbox{backcolour}{\texttt{#1}}}

\newcommand{\imgcaption}[2]{\tiny \textbf{Figura #1.} #2.}

\newcommand{\mgc}[2][]{\colorbox{backcolour}{\texttt{\_\_#2\_\_#1}}}

\newcommand{\mgccapt}[1]{\texttt{\_\_#1\_\_}}

% Hyperlinks configurations
\hypersetup{
	colorlinks=true,
	linkcolor=black,
	filecolor=magenta,      
	urlcolor=cyan,
	pdftitle={Overleaf Example},
	pdfpagemode=FullScreen,
}

%----------------------------------------------------------------------------------------

\begin{document}
	% Print the title
	\maketitle
	
	%----------------------------------------------------------------------------------------
	%	ARTICLE CONTENTS
	%----------------------------------------------------------------------------------------
	
	
	\section{Repositorio del proyecto}
	
	\begin{center}
		\href{https://github.com/science-engineering-art/traffic-lights}{https://github.com/science-engineering-art/traffic-lights}
	\end{center}
	
	\section{Descripción}
	
	En todo el mundo, la congestión del tráfico sigue siendo un problema importante en la mayoría de las ciudades, debido al creciente número de vehículos privados,  de mercancías y de transporte público. Este fenómeno afecta, sobre todo en horas pico, a los usuarios de la red vial, los cuales pierden mucho tiempo en la carretera;  además de incidir de manera negativa en el medio ambiente pues los carros se encuentran más tiempo encendidos liberando gases a la atmósfera.
	
	\vspace{0.5em}
	Se puede pensar en varias soluciones para este problema:
	\begin{enumerate}
		\item Construcción de nuevas carreteras: Muchas veces esto no es posible debido a las condiciones geográficas, y más importante aún, es muy costoso, por lo que en general no es una solución viable.
		\item Mejora del sistema de señalización vial: Es más sensata pues se relaciona inteligentemente con la infraestructura existente. Es de especial interés la mejora de los semáforos ya que estos controlan el flujo de la red vial de la ciudad. En estos podemos tener:
		\begin{itemize}
			\item Plan de luces fijo (Estático): se fijan los tiempos de verde y rojo en cada línea de luces de una intersección así como su secuencia una sola vez teniendo en cuenta las previsiones de tráfico, y estas no cambian.
			\item Controladores de tiempo real (Dinámicos): en técnicas de tiempo real, el sistema debe ser capaz de adaptarse inmediatamente (o muy brevemente) a las condiciones del tráfico. Dicho sistema posee algoritmos que permiten controlar el tráfico, los cuales reciben información sobre el estado del tráfico, que ha sido recolectada por los sensores colocados en cada carril, y reacalculan la duración y la sincronización de la luces para minimizar la congestión, es decir, para minimizar el tiempo promedio de espera en las luces, y la duración de colas.
		\end{itemize} 
	\end{enumerate}

	\subsection{Objetivo}
	
	Creación de un algoritmo de control que determine de la secuencia de fases y el tiempo de de luz verde óptimos en los semáforos de las intersecciones, con el fin de hacer más fluido el tráfico y minimizar las colas.
	
	\section{Simulación}
	
	\subsection{Modelo mesosc\'opico}
	
	Para modelar el flujo del tráfico se utiliza un modelo mesosc\'opico, modelo híbrido que combina las características de los modelos microscópico y macroscópico.
	
	Como en el modelo microsc\'opico, se representan los vehículos de forma independiente y se intenta replicar el comportamiento de un conductor. En consecuencia, es un sistema multiagente, es decir, cada vehículo opera por sí mismo utilizando información de su entorno.
	
	Cada vehículo est\'a identificado por un número i. El i-ésimo vehículo sigue al (i-1)-ésimo vehículo. Para el i-ésimo vehículo, se denota por $ x_{i} $ su posición a lo largo del camino, $ v_{i} $ su velocidad y $ l_{i} $ su longitud. Sean, adem\'as, $ s_{i} $ la distancia parachoques a parachoques y $ \Delta v_{i} $ la diferencia de velocidad entre el i-ésimo vehículo y el vehículo que le precede (vehículo número i-1).
	
	\begin{center}
		\includegraphics[width=\columnwidth]{microscopic_model.png}
	\end{center}
	
	\begin{center}
		$ s_{i} = x_{i} - x_{i-1} - l_{i} $
	
		$ \Delta v_{i} = v_{i} - v_{i-1} $
	\end{center}
	
	Por otro lado, como caracter\'istica del modelo macrosc\'opico, se tienen en cuenta factores globales que describen el movimiento de vehículos como un todo, en términos de densidad de tráfico (vehículos por km) y flujo de tráfico (vehículos por minuto).
	
	\todo{revisar y arreglar como se utiliza el modelo macroscopico}
	
	\subsection{Modelo Conductor Inteligente}
	
	El Modelo de Conductor Inteligente describe la aceleración del i-ésimo vehículo en función de sus variables y las del vehículo que le precede. La ecuación dinámica se define como:
	
	\begin{center}
		$  \dfrac{dv_{i}}{dt} = a_{i} \left(1 - \left(\dfrac{v_{i}}{v_{0,i}}\right)^{\delta} - \left(\dfrac{s^{*}(v_{i}, \Delta v_{i})}{s_{i}}\right)^{2}\right)$
		
		\vspace{0.5em}
		$ s^{*}(v_{i}, \Delta v_{i}) =  s_{0,i} + v_{i}T_{i} + \dfrac{(v_{i} \Delta v_{i})}{\sqrt{(2a_{i} b_{i})}}$
	\end{center}
	donde:
	\begin{itemize}
		\item $ s_{0,i} $ : es la distancia mínima deseada entre el vehículo i e i-1.
		\item $ v_{0,i} $ : es la velocidad máxima deseada del vehículo i.
		\item $\delta$ : es el exponente de la aceleración y controla la “suavidad” de la aceleración.
		\item $ T_{i} $  : es el tiempo de reacción del conductor del i-ésimo vehículo.
		\item $ a_{i} $ : es la aceleración máxima del vehículo i.
		\item $ b_{i} $ : es la desaceleración cómoda para el vehículo i.
		\item $ s^{*} $ : es la distancia real deseada entre el vehículo i e i-1.
	\end{itemize}
	
	Interpretando los t\'erminos en $ s^{*} $:
	\begin{itemize}
		\item $ v_{i}T_{i}  $ : es la distancia de seguridad del tiempo de reacción. Es la distancia que recorre el vehículo antes de que el conductor reaccione (frene).
		Dado que la velocidad es la distancia entre el tiempo, la distancia es la velocidad por el tiempo.
		\item $ (v_{i} \Delta v_{i})/\sqrt{(2a_{i} b_{i})} $: es una distancia de seguridad basada en la diferencia de velocidad. Representa la distancia que tardará el vehículo en reducir la velocidad (sin chocar con el vehículo de delante), sin frenar demasiado (la deceleración debe ser inferior a $ b_{i} $).
	\end{itemize}
	
	\subsection{Modelo de red vial de tráfico}
	
	Para modelar la red vial se utiliza un grafo dirigido, donde las aristas representan las calles (cuadras) y los vértices las intersecciones. Cada vehículo tiene un camino que consta de múltiples calles. Se aplica el Modelo Conductor Inteligente para vehículos en la misma calle. Cuando un vehículo llega al final de la calle, se retira de la misma y es añadido a la siguiente.
	
	\subsection{Generación de vehículos}
	
	Los veh\'iculos se generan en los extremos del mapa. Cada calle tiene un $\lambda$ asignado que representa la cantidad de carros por segundo que pasan por ella, el cual corresponde a un valor entre 70 y 150 carros por hora. A las calles que se determinan como principales se les aumenta su $\lambda$.
	
	Se usa $\lambda$ para saber la probabilidad de que se genere un carro en dicha calle en un intervalo de tiempo, utiliz\'andolo como par\'ametro en la funci\'on de probabilidad de Poisson. Esta funci\'on dir\'a cu\'al es la probabilidad de que hayan pasado $ x $ carros en un intervalo de tiempo en esa calle.  
	
	\begin{center}
		$ p(x) = f(x, \lambda) = \dfrac{\lambda^{x} \cdot e^{-\lambda}}{k!} $		
	\end{center}

	Luego, se genera un n\'umero $ r $ aleatorio entre $ 0 $ y $ 1 $, y se halla la probabilidad $ p(0) $ de que no se haya creado ning\'un carro en este intervalo de tiempo. Se genera un carro en esta calle si $ r > p(0) $.
	
	Al generar un carro se establece su destino y su ruta. El destino se selecciona aleatoriamente de acuerdo a una probabilidad que tiene cada calle de que un carro vaya hacia ella, asign\'andole mayor peso a las calles m\'as transitadas. Cuando se ejecuta por primera vez una simulaci\'on en un mapa, se utiliza Floyd-Warshall para hallar los caminos de menor distancia entre todo par de calles, y se guardan los mismos. Luego, se escoge la ruta de estos caminos precalculados. 
	
	%Para la generación de vehículos se utiliza la distribución de Poisson. Esta es una distribución de probabilidad discreta, describe situaciones en las cuales los clientes llegan de manera independiente durante un cierto intervalo de tiempo y el número de llegadas depende de la magnitud del intervalo.
	
% 	\begin{center}
%		$ p(x) = f(x, \lambda) = \dfrac{\lambda^{x} \cdot e^{-\lambda}}{k!} $		
% 	\end{center}
 
	\subsection{Semáforos}
	
	\subsubsection{Turnos}
	
	Un turno es un conjunto de sem\'aforos que se encuentran en verde en un mismo per\'iodo de tiempo. Cada sem\'aforo de la intersecci\'on puede existir en m\'as de un turno, siendo posible as\'i establecer distintas combinaciones para las direcciones factibles (seguir recto, doblar a la derecha o doblar a la izquierda) tal que no haya choques y haya la menor cantidad de turnos posibles. 
	
	\subsubsection{Zonas}
	
	Los sem\'aforos tienen dos zonas en los que los veh\'iculos se comportan de forma diferente:
	\begin{itemize}
		\item Zona de ralentización: Zona en la que los vehículos reducen su velocidad máxima utilizando un factor de ralentización.
		\begin{center}
			$ v_{0,i} := \alpha v_{0,i} \hspace{0.5em} \text{donde} \hspace{0.5em} \alpha < 1$
		\end{center}
		\item Zona de parada: Zona en la que se detienen los vehículos. Esto se logra utilizando una fuerza de amortiguamiento a través de la siguiente ecuación dinámica:
		\begin{center}
			$ \dfrac{dv_{i}}{dt} = -b_{i} \dfrac{v_{i}}{v_{0,i}} $
		\end{center}
	\end{itemize}
	
	\subsection{Curvas de Bézier}
	
	Una curva de Bézier es una curva polinomial que aproxima a una serie de puntos llamados puntos de control. Est\'a definida por un conjunto de puntos de control $ P_{0} $ a $ P_{n} $ donde $ n $ es su grado. Se dice que una curva de grado n aproxima a $ n + 1 $ puntos de control. El primer y el último punto de control son siempre los puntos extremos de la curva; sin embargo, los puntos de control intermedios (si los hay) por lo general no se encuentran en la misma. 
	
	\begin{center}
		$ P(u) = \sum_{i=0}^{n} P_{i}B_{i}^{n}(u) $
	\end{center}
	\begin{center}
		$ B_{i}^{n}(u) = \binom{n}{i} (1 - u)^{n-i}u^{i} $
	\end{center}
	donde $ P_{i} $ es el conjunto de puntos, $ B_{i}^{n}(u) $ representa los polinomios de Bernstein y $ u $ toma valor entre $ 0 $ y $ 1 $.

	Para producir una curva en nuestro mapa se crean las calles (rectas) y se utilizan las curvas de B\'ezier como un spline de suavizado. N\'otese que esto solo se utiliza para la parte visual, en realidad los carros estar\'an corriendo en una sola calle. 
	
	\section{Inteligencia Artificial}
	
	\subsection{Algoritmo genético}
	
	\subsubsection{Optimizaci\'on}
	
	Mediante el uso de un algoritmo gen\'etico se intenta minimizar las colas demoradas de carros en las intersecciones.
	
	\subsubsection{Solución}
	
	La solución será un vector que contiene enteros correspondientes a los tiempos de luz verde de cada turno de la intersección, para todas las intersecciones del mapa.
	
	\subsubsection{Población inicial}
	
	La población inicial se hallará de manera aleatoria, en donde cada uno de los valores de tiempo de luz verde estará entre el tiempo promedio que le toma a un carro pasar por la intersección y el tiempo máximo que un vehículo puede estar esperando. 
	
	\subsubsection{Función de evaluación \textit{fitness}}
	
	Para evaluar qu\'e tan buena es una soluci\'on, se ejecuta con ella una simulaci\'on. Luego, con los resultados arrojados por esta en un per\'iodo determinado y bajo el criterio escogido, se le da una puntuaci\'on a dicho individuo.
	
	Se consideraron tres formas de calcular el fitness:
	\begin{itemize}
		\item Seg\'un el m\'aximo tiempo de espera. 
		\item Seg\'un el tiempo total de paso de los carros por una calle. 
		\item Seg\'un la media ponderada, calculada sobre todas las calles, del tiempo promedio que toma a los carros cruzar una calle.
	\end{itemize}

	N\'otese que como se quieren minimizar los tiempos de los criterios anteriores, se tomar\'an dichos valores negativos para el fitness. 
	
	\subsubsection{Selección de padres}
	
	Para escoger los padres se utiliz\'o la selección basada en el rango (\textit{rank selection}), en la cual se ordena según el valor de fitness y se da una probabilidad de selección a cada individuo. 
	
	La selección basada en el rango reduce los efectos potencialmente dominantes de individuos de alto fitness, comparativamente, en la población, estableciendo una cantidad predecible y limitada de presión de selección a favor de tales individuos. Al mismo tiempo, exagera la diferencia entre valores de fitness agrupados de forma cercana para que los mejores se puedan muestrear más.
	
	\subsubsection{Mutación}
	
	La funci\'on de mutaci\'on recibe una poblaci\'on y una probabilidad de mutaci\'on igual al inverso del n\'umero de individuos. Por cada gen de cada individuo, se escoge un n\'umero $ m $ aleatorio entre $ 0 $ y $ 1 $, y dicho gen es mutado si $ m $ es menor que la probabilidad de mutaci\'on. La forma de mutar es escoger un n\'umero aleatorio entre el tiempo promedio que le toma a un carro pasar por la intersección y el tiempo máximo que un vehículo puede estar esperando, de igual forma que para crear la poblaci\'on inicial.
	
	\subsubsection{Cruzamiento}
	
	La funci\'on de cruzamiento genera dos individuos para a\~nadir a la nueva poblaci\'on a partir de los padres. Se consideraron tres formas de realizar el cruzamiento:
	\begin{itemize}
		\item  \textbf{Cruzamiento por puntos m\'ultiples:} Se seleccionan p puntos random y los segmentos (resultado de picar los individuos por dichos puntos) alternos de los individuos se intercambian para obtener nuevos descendientes.
		\item \textbf{Cruzamiento geom\'etrico:} Se emparejan los genes que tienen la misma posici\'on en los padres, y en esta posici\'on del hijo se pone la media geom\'etrica entre ellos. Adem\'as de este descendiente, se devuelve uno de los padres, escogido de forma aleatoria.   
		\item \textbf{Cruzamiento intermedio:} Se emparejan los genes que tienen la misma posici\'on en los padres, y en esta posici\'on del hijo se pone la media aritm\'etica entre ellos. Adem\'as de este descendiente, se devuelve uno de los padres, escogido de forma aleatoria. 
	\end{itemize}
	
	\subsection{$ A^{*} $}
	
	La b\'usqueda $ A^{*} $ es un algoritmo de búsqueda \textit{best-first} informada que utiliza la función de evaluación:
	\begin{center}
		$ f(n) = g(n) + h(n) $
	\end{center}
	donde $ g(n) $ es el costo de alcanzar el nodo $ n $, $ h(n) $ es una heur\'istica del costo estimado del camino de menor costo desde n hasta el objetivo, siendo $ f(n) $ entonces el costo estimado de la mejor soluci\'on que pasa por $ n $.
	
	Se concibieron dos algoritmos $ A^{*} $, el cl\'asico para hallar la distancia m\'inima entre dos calles, y otro para hallar el camino que menor tiempo le tomar\'a a un veh\'iculo para llegar de una calle a otra.
	
	\subsubsection{$ A^{*} $ para hallar la distancia m\'inima entre dos calles}
	
	\begin{center}
	
	\textit{$ g(n) = $ distancia recorrida hasta el momento.}
	
	\vspace{0.5em}
	\textit{$ h(n) = $ distancia en l\'inea recta desde el punto actual hasta el objetivo.}
	
	\end{center}
	
	\subsubsection{$ A^{*} $ para hallar el camino que menor tiempo le tomar\'a a un veh\'iculo para llegar de una calle a otra}
	
	\begin{center}
		\textit{$ g(n) = $ } \textit{tiempo que demor\'o el veh\'iculo en llegar desde el estado inicial hasta el \hspace{8em} estado intermedio n.}
	\end{center}	
	Para hallar $ g(n) = $ se corre una simulaci\'on con un veh\'iculo que vaya desde el estado inicial hasta el estado intermedio n, y se toma el tiempo que demor\'o en llegar a su destino. \todo{aclarar lo de que va soltando carros en cada esquina}
		
	\begin{center}
		\vspace{0.5em}
		\textit{$ h(n) = $ } \textit{aproximado del tiempo requerido para llegar del estado intermedio n \hspace{8em} hasta el estado final.}
	\end{center}

	Se precacula, ejecutando Disjktra desde el destino hacia todos los puntos del mapa, el tiempo que tomar\'a cada una de esas rutas, teniendo en cuenta los sem\'aforos, distancia de las calles, velocidad de los carros y tr\'afico. Luego, para hallar $ h(n) $ simplemente se toma el tiempo de la ruta del destino al nodo intermedio n.
	

	\section{Tests}
	
	Se ejecutaron \todo{chequear cantidad de tests}50 pruebas con los siguientes par\'ametros para la optimizaci\'on:
	\begin{itemize}
		\item \textbf{tama\~no de la poblaci\'on:} 30
		\item \textbf{cantidad de iteraciones:} 50
		\item \textbf{tiempo de observaci\'on de cada individuo en la simulaci\'on:} 10s
		\item \textbf{velocidad de la simulaci\'on:} 30
		\item \textbf{tipo de cruzamiento:} cruzamiento por puntos m\'ultiples (2)
		\item \textbf{c\'alculo de fitness:} seg\'un la media ponderada, calculada sobre todas las calles, del tiempo promedio que toma a los carros cruzar una calle
		\item \textbf{tiempo promedio que le toma a un carro pasar por la intersección:} 3s
		\item \textbf{tiempo máximo que un vehículo puede estar esperando:} 90s
	\end{itemize}
	sobre el mapa, de $ 12 \times 6 $ cuadras y $ 29 $ intersecciones semaforizadas, que se muestra a continuaci\'on:
	
	\begin{center}
		\includegraphics[width=\columnwidth]{map.jpg}
	\end{center}
	
	Dichas pruebas se pueden encontrar en la carpeta \emph{tests} del repositorio. 

	\subsection{Resultados}

	\missingfigure{plot con los fitness de la solucion final en cada test}
	\todo{Explicar resultados}
	
	\section{Recomendaciones}
	
	\todo{Poner algun problema actual}
	
	\todo{Trabajo futuro: creacion de los mapas con gramatica prob. como habia pensado leo}
	
	Como trabajo futuro ser\'ia interesante optimizar la distribuci\'on de los turnos tal que en una intersecci\'on haya la menor cantidad posible de estos, y optimizar la posici\'on de las intersecciones semaforizadas (a\~nadir, quitar o reacomodar intersecciones en el mapa), para contribuir a hacer el tr\'afico m\'as fluido, objetivo perseguido en este proyecto.
	
	\listoftodos
\end{document}


